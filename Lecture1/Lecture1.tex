\documentclass[10pt,a4paper]{article}
\usepackage[utf8]{inputenc}
\usepackage[russian]{babel}
\usepackage{amsthm}

\newtheorem{definition}{Определение}
\newtheorem{theorem}{Теорема}
\newtheorem{corollary}{Следствие}

\usepackage{amssymb}
\newcommand*{\Z}{\mathbb{Z}}


\begin{document}
	% Изменение отсчета глав, чтобы глава начиналась с 2
	\setcounter{section}{1}
	\section{Теория колец}
		\subsection{Введение}
			\begin{definition}
				Кольцо $R$ - множество элементов с двумя бинарными операциями, такие что:
				\begin{enumerate}
					\item $a + b = b + a$ (коммутативность)
					\item $(a + b) + c = a + (b + c)$ (ассоциативность)
					\item $\exists$ нейтральный элемент относительно сложения:\\
					$a+0 = 0+a = a, \forall a \in R$
					\item $\exists$ противоположный отностильно сложения:\\
					$-a \in R \big| (-a) + a = a + (-a) = 0$
					\item ассоциативность\\
					$a(bc) = (ab)c$
					\item дистрибутивность\\
					$a(b+c) = ab + ac$\\
					$(b+c)a = ba + ca$
				\end{enumerate}
			\end{definition}
			\begin{theorem}[Свойства колец]
				Пусть $a, b, c \in R$ - кольцо
				\begin{enumerate}
					\item $a \cdot 0 = 0 \cdot a = 0$
					\item $a(-b) = (-a)b = -(ab)$
					\item $(-a)(-b) = ab$
					\item $a(b-c) = ab - ac$\\
					$(b-c)a = ba - ca$
					\\\\ Кроме того, если $R$ имеет единичный элемент $1$ относительно умножения, то
					\item $(-1)a = -a$
					\item $(-1)(-1) = 1$
				\end{enumerate}
			\end{theorem}
			\begin{theorem}
				Если кольцо имеет единичный элемент, то этот элемент единственен. Если для $a \in R \exists a^{-1}$, то $a^{-1}$ - единственен.
			\end{theorem}
			\begin{definition}
				Подмножество $S$ кольца $R$ называется подкольцом в $R$, если само является кольцом относительно операции, заданных в $R$.
			\end{definition}
			\begin{theorem}[Признак подкольца]
				Непустое подмножество $S$ кольца $R$ является подкольцом, если $S$ замкнуто относительно операций минус и умножить, т.е. если
				\begin{tabular}{l}
				$a-b$\\
				$ab$			
				\end{tabular}
				$\Big\} \in R, \forall a, b \in S$
			\end{theorem}
		\subsection{Кольца целостности}
			\begin{definition}
				Делитель нуля $0 \neq a \in R$ - коммутативное кольцо $\big| \exists 0 \neq b \in R $ и $ab = 0$.
			\end{definition}
			\begin{definition}
				Кольцо целостности - коммутативное кольцо с единицей без делителей нуля.
			\end{definition}
			\begin{theorem}
				Пусть $a, b, c \in R$ - кольцо целостности. Если $a \neq 0$ и $ab = ac$, то $b = c$.
			\end{theorem}
			\begin{proof}
				Рассмотрим $ab = ac$\\
				$ab - ac = 0$\\
				$a(b-c) = 0$, где $a \neq 0$\\
				$\Rightarrow b - c = 0$\\
				$b = c$
			\end{proof}
			\begin{definition}
				Поле - коммутативное кольцо с единицей, в котором любой отличный от нуля элемент обратим.
			\end{definition}
			\begin{theorem}
				Конечное кольцо целостности является полем.
			\end{theorem}
			\begin{proof}
				Пусть $D$ - конечное кольцо целостности с единицей.\\
				Рассмотрим $0 \neq a \ in D$\\
				Докажем, что $a$ - обратим\\
				Если $a = 1 \Rightarrow$ очевидно\\
				Пусть $a \neq 1$\\
				\begin{tabular}{l}
				$a, a^2, a^3, ...$\\
				$D$ - конечно			
				\end{tabular}
				$\Big\} \Rightarrow \exists i > j \big| a^i = a^j$\\
				$\Rightarrow a^i - a^j = 0$\\
				$a^j(a^{i-j}-1) = 0$, где $a^j \neq 0$\\
				$a^{i-j} = 1$\\
				$\Rightarrow a^{i-j-1}$ - обратный к $a$.
			\end{proof}
			\begin{corollary}
				Для $\forall p$ - простых, $\Z_p$ - поле.
			\end{corollary}
			\begin{definition}
				Характеристика кольца $R$ - наименьшее положительное целое $n \big| n \cdot x = 0, \forall x \in R$.\\
				Если такого $n$ не существует, то будем говорить, что $R$ имеет характеристику 0. Обозначается $char R = n$.
			\end{definition}
			\begin{theorem}
				Пусть $R$ - кольцо с единицей. Если единица имеет бесконечный порядок относительно сложения, то $char R = 0$. Если единица имеет порядок $n$ относительно сложения, то $char R = n$.
			\end{theorem}
			\begin{theorem}
				Если $R$ - кольцо целостности, то $char R = \Big[$
				\begin{tabular}{l}
				0\\
				p - простое
				\end{tabular}
			\end{theorem}
\end{document}
