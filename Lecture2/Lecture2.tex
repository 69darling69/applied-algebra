\documentclass[10pt,a4paper]{article}
\usepackage[utf8]{inputenc}
\usepackage[russian]{babel}
\usepackage{amsthm}

\newtheorem{definition}{Определение}
\newtheorem{theorem}{Теорема}
\newtheorem{corollary}{Следствие}

\usepackage{amssymb}
\newcommand*{\Z}{\mathbb{Z}}


\begin{document}
	% Изменение отсчета глав, чтобы глава начиналась с 2
	\setcounter{section}{1}
	\section{Теория колец}
		\setcounter{subsection}{2}
		\subsection{Идеалы и фактор-кольца}
			\paragraph{Идеалы}
				\begin{definition}
					Полькольцо $A$ кольца $R$ называется идеалом кольца $R$, если для $\forall r \in R$ и $\forall a \in A : ra; ar \in A$.\\
					Идеал $A$ в $R$ называется собственным идеалом кольца $R$, если $A \subset R$. 
				\end{definition}
			\paragraph{Фактор-кольца}
				Пусть  $R$ - кольцо $\Rightarrow R$ - аддитивная группа; $A$ - идеал в $R \Rightarrow$ рассмотрим ее как нормальную подгруппу.
				$\Rightarrow {}^R/_A = \{r + A | r \in R\}$.
				\begin{theorem}
					Пусть $R$ - кольцо; $A$ - полькольцо в $R$.
					Множество классов $\{r + A | r \in R\}$ - кольцо относительно операций:\\
					\begin{tabular}{l}
						$(s + A) + (t + A) = s + t + A$\\
						$(s + A) \cdot (t + A) = st + A$
					\end{tabular}
					$\Leftrightarrow A$ - идеал в $R$.
				\end{theorem}
\end{document}
