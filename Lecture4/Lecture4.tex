\documentclass[10pt,a4paper]{article}
\usepackage[utf8]{inputenc}
\usepackage[russian]{babel}
\usepackage{amsthm}

\newtheorem{definition}{Определение}
\newtheorem{theorem}{Теорема}
\newtheorem{corollary}{Следствие}

\usepackage{amssymb}
\newcommand*{\Z}{\mathbb{Z}}


\begin{document}
	% Изменение отсчета глав, чтобы глава начиналась с 2
	\setcounter{section}{1}
	\section{Теория колец}
		\setcounter{subsection}{3}
		\subsection{Гомоморфизм колец}
			\subsubsection{Основные определения и примеры}
				\begin{definition}
					Гомоморфизмом колец $\varphi: R \to S$ (кольца) называется отображение:
					\begin{enumerate}
						\item $\varphi(a+b) = \varphi(a) + \varphi(b)$
						\item $\varphi(ab) = \varphi(a) \cdot \varphi(b)$
					\end{enumerate}
					Изоморфизмом колец называется гомоморфизм колец, действующий биективно.
				\end{definition}
			\subsubsection{Свойства гомоморфизмов колец}
				\begin{theorem}
					Пусть $\varphi: R \to S$ - гомоморфизм колец, $A$ - подкольцо в $R; B$ - идеал в $S$.
					\begin{enumerate}
						\item Для $\forall r \in R$ b положительного целого $n:$
							\begin{tabular}{l}
								$\varphi(n \cdot r) = n \cdot \varphi(r)$\\
								$\varphi(r^n) = (\varphi(r))^n$
							\end{tabular}
						\item $\varphi(A) = \{\varphi(a) \big| a \in A\}$ - подкольцо в $S$
						\item $\varphi^{-1}(B) = \{r \in R \big| \varphi(r) \in \varphi\}$ - идеал в $B$
						\item Если $R$ - коммутативно, то $\varphi(R)$ - коммутативно
						\item Если $1 \in R; S \neq \{0\}$ и $\varphi$ - сюръективно, то $\varphi(1)$ - обратим (единица) в $S$
						\item $\varphi$ - изоморфизм $\Leftrightarrow \varphi$ - сюръективно и $Ker\ \varphi = \{r \in R \big| \varphi(r) = 0\} = \{0\}$
						\item Если $\varphi$ - изоморфизм, то $\varphi^{-1}$ - изоморфизм
					\end{enumerate}
				\end{theorem}
				\begin{theorem}
					Пусть $\varphi: R \to S$ - изоморфизм колец. Тогда $Ker\ \varphi = \{r \in R \big| \varphi(r) = 0\}$ - идеал в $R$
				\end{theorem}
				\begin{theorem}
					Пусть $\varphi: R \to S$ - изоморфизм колец. Тогда отображение
					\begin{tabular}{l}
						${}^R/_{Ker\ \varphi} \to \varphi(R)$\\
						$r + Ker\ \varphi \to \varphi(r)$
					\end{tabular}
					- изоморфизм: ${}^R/_{Ker\ \varphi} \cong \varphi(R)$
				\end{theorem}
				\begin{theorem}
					$\forall$ идеал в кольце $R$ является ядром гомоморфизма кольца $R$. В частности, идеал $A$ из $R$ есть $Ker\ \varphi$, где $\varphi:$
					\begin{tabular}{l}
						$R \to {}^R/_A$\\
						$r \to r+A$
					\end{tabular}
				\end{theorem}
				\begin{theorem}
					Пусть $R$ - кольцо с единицей.\\
					Отображение $\varphi:$
					\begin{tabular}{l}
						$\Z \to R$\\
						$n \to n \cdot 1$
					\end{tabular}
					- гомоморфизм колец
				\end{theorem}
				\begin{corollary}
					Если $R$ - кольцо с единицей и $char(R) = n > 0$, то $R$ содержит подкольцо, изоморфное $\Z_n$\\
					Если $char(R) = 0$, то $R$ содержит подкольцо, изоморфное $\Z$.
				\end{corollary}
				\begin{proof}
					Пусть $1$ - единица $R$\\
					Рассмотрим $S = \{k \cdot 1 \big| k \in Z\}$\\
					По предыдущей теореме: отображение $\varphi:$
					\begin{tabular}{l}
					$\Z \to S$\\
					$k \to k \cdot 1$
					\end{tabular}
					- гомоморфизм колец\\
					По первой теореме об изоморфизмах: ${}^{\Z}/_{Ker\ \varphi} \cong \varphi(\Z)$\\
					$Ker\ \varphi = \{k \in \Z \big| \varphi(k) = 0\}$, где $\varphi(k) = k \cdot 1 \Rightarrow k \cdot 1 = 0 \Rightarrow k \ char\ S \Rightarrow k$ - аддитивный порядок $1 \Rightarrow Ker\ \varphi = (k)$\\
					${}^{\Z}/_{Ker\ \varphi} = {}^{\Z}/_{(k)} = \Z_k$\\
					$\varphi(\Z) \subset S$\\
					$\Z_k \cong \varphi(\Z) \subset S$\\
					Можно рассмотреть отображение на себя (сюръективность)\\
					$k = char\ \Z_k = char\ S \Rightarrow \varphi(\Z) = S \Rightarrow \Z_k \cong S$\\
					Если $char\ R = 0 \Rightarrow S \cong {}^{\Z}/{(0)} \cong \Z$
				\end{proof}
				\begin{corollary}
					Для $\forall$ положительныц целых $m$ отображение $\varphi:$
					\begin{tabular}{l}
						$\Z \to \Z_m$\\
						$x \to x\ (mod\ m)$
					\end{tabular}
					является гомоморфизмом.
				\end{corollary}
				\begin{corollary}
					Пусть $F$ - поле и $char\ F = p$. Тогда $F$ содержит подполе, изоморфное $\Z_p$. Если $char\ F = 0$, то $F$ содержит подполе, изоморфное полю рациональных чисел.
				\end{corollary}
				\begin{theorem}
					Пусть $D$ - кольцо целостности. Тогда $\exists$ поле $F$, которое содержит подкольцо, изоморфное $D$.
				\end{theorem}
\end{document}
